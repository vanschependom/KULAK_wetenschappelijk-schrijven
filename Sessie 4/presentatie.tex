\documentclass
[kulak] % [kulak|kul|kuleuven-brugge,handout|outline|]
{kulakbeamer}

\usepackage[dutch]{babel}
\usepackage[T1]{fontenc}
\usepackage{graphicx}

\title[MEBP]{Mijn eerste Beamer-presentatie}
\subtitle{Ondertitel}
\author[VVS]{Vincent Van Schependom} 
\institute[Kulak]{KU Leuven Kulak}
\date{Academiejaar 2023 -- 2024}

\AtBeginSection[]{
	\begin{frame}
		\frametitle{Overzicht}
		\hfill \large\parbox{.95\textwidth}{\tableofcontents[currentsection,hideothersubsections]}
	\end{frame}
}

\renewcommand{\outlineTitle}{Overzicht} % enkel met optie outline

\begin{document}
	
	\begin{titleframe}
		\titlepage
	\end{titleframe}
	
	% Outlineframe heeft donkere achtergrond
	\begin{outlineframe}[Overzicht]
		\tableofcontents
	\end{outlineframe}
	
	% % % Here you go  % % % 
	
	\section{Inleiding}
	
	% Gewone frame heeft lichte achtergrond
	\begin{frame}
		\frametitle{Inleidende frame}
		Inleidende tekst.
	\end{frame}
	
	\section[Korte titel]{Lange hoofdstuktitel}
	
	\begin{frame}
		\frametitle{Frame-titel}
		
		\begin{block}{Priemgetal}
			Een natuurlijk getal is een priemgetal als het precies twee delers heeft, één en zichzelf.
		\end{block}
		
		\begin{alertblock}{Een}
			Het getal één is geen priemgetal.
		\end{alertblock}
		
		\begin{exampleblock}{Twee}
			Twee is een heel bijzonder priemgetal (an odd prime).
		\end{exampleblock}
	\end{frame}
	
	\begin{frame}{Kolommen}
	
		\begin{columns}
			\column{.5\textwidth}
			\begin{figure}
				\includegraphics[width=.5\columnwidth]{vincent}
			\end{figure}
			\column{.5\textwidth}
			\begin{itemize}
				\item Vincent Van Schependom
				\item Student informatica
				\item Geen West-Vlaming
			\end{itemize}
		\end{columns}
	
	\end{frame}
	
	\section{Pauzeren}
	
	\begin{frame}{Special effects}
		Dit staat er eerst
		
		\pause
		
		Dit staat er pas later
		
		\pause
		
		Dit staat er nog later
	\end{frame}
	
	\section{Overlays}
	
	\begin{frame}{Speciale types pauzes}
		Dit staat er eerst
		
		\uncover<4>{Dit staat er later (uncover) en er wordt altijd plaats vrijgehouden hiervoor}
		
		\only<3>{Dit staat er niet altijd (only) en er wordt \textbf{geen} plaats vrijgehouden}
		
		\alert<2>{Dit wordt benadrukt (alert)}
	\end{frame}
	
	\begin{frame}{Verschrikkelijke itemize pauzes}
		De \texttt{<} en \texttt{>} tekens zijn voor de slidespecificaties.
		\begin{itemize}
			\item<1-> Een
			\item<2-> Twee \hypertarget<2>{targetlabel2}{}
			\item<3-> Drie
		\end{itemize}
	\end{frame}
	
	\begin{frame}{Geautomatiseerde verschrikkelijke itemize pauzes}
		We kunnen ook aan de \texttt{itemize} een argument meegeven waarin we de slidespecificaties definiëren, bijvoorbeeld \texttt{[<+->]}
		\begin{itemize}[<+->]
			\item Een
			\item Twee \hypertarget{targetlabel}{X}
			\item Drie
		\end{itemize}
	\end{frame}
	
	\section{Kruisverwijzingen}
	
	\begin{frame}{Hypertarget}
		\begin{itemize}
			\item We willen verwijzen naar een specifiek frame op een specifiek moment, wanneer een bepaalde bullet point bijvoorbeeld nog niet getoond is.
			
			\item Dit kan met \texttt{hypertarget} en \texttt{hyperlink}.
			
			\item \hyperlink{targetlabel}{Dit} gaat bijvoorbeeld naar het tweede item van de automatische.
			
			\item \hyperlink{targetlabel2}{Dit} gaat naar het tweede item van de handmatige.
		\end{itemize}
	\end{frame}
	
	\section{Besluit}
	\begin{frame}
		\frametitle{Afsluitende frame}
		Afsluitende tekst.
	\end{frame}
	
\end{document}
