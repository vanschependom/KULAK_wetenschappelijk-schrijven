\documentclass
[kulak] % [kulak|kul|kuleuven-brugge,handout|outline|]
{kulakbeamer}

\usepackage[dutch]{babel}
\usepackage[T1]{fontenc}
\usepackage{graphics}
\usepackage{siunitx}

\setbeamertemplate{footline}{}

\begin{document}
	
	\begin{frame}
		\frametitle{Vincent Van Schependom - ik en mijn favoriete formule}
		\begin{columns}
			\column{0.5\textwidth}
			\begin{figure}
				\vspace*{-0.7cm}
				\includegraphics[width=.4\columnwidth]{vincent}
			\end{figure}
			Student informatica
			\begin{itemize}
				\item Instagram: \href{https://instagram.com/schependom}{@schependom}
				\item LinkedIn: \href{https://www.linkedin.com/in/vanschependom/}{linkedin.com/in/vanschependom}
				\item Mail: \\ \href{mailto:vincent.vanschependom@student.kuleuven.be}{ vincent.vanschependom@student.kuleuven.be}
			\end{itemize}
			\column{0.5\textwidth}
			\textbf{Het vermogen (of \textit{wattage})}
			\[P=\frac{dW}{dT}\]
			Een heel belangrijke formule voor wielrenners, zoals ik.
			\begin{itemize}
				\item W is de arbeid, met \(W=\int\vec{F}\cdot d\vec{r}\)
				\item T is een tijdsinterval
				\item Eenheid: \si{J/s = W}
			\end{itemize}
		\end{columns}
	\end{frame}
	
\end{document}
