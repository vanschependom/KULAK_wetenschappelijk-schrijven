\documentclass{article}

\usepackage[dutch]{babel}
\usepackage{hyperref}
\usepackage{graphicx}
\usepackage{amssymb, amsthm, amsmath}

\title{Ik en mijn favoriete formule}
\author{Vincent Van Schependom
	\\ \textit{Student Informatica}}
\date{}

\begin{document}
	
	\maketitle
	
	\begin{figure}[h]
		\centering
		\includegraphics[width=5cm]{vincent}
	\end{figure}
	
	Ik ben \textbf{Vincent Van Schependom}, student informatica. Ik zit op kot in residentie \textit{De Corona}. In mijn vrije tijd ben ik een fervente sporter; wielrennen is mijn grote passie, maar daarnaast ga ik ook regelmatig lopen. Om een centje bij te verdienen, maak ik logo's, websites en dergelijke meer voor klanten. Dit doe ik via de zaak van mijn mama, die zelf ook programmeur is. Als ik in mijn drukke agenda nog een gaatje vind, hou ik me ook graag bezig met fotografie. Je kan me via onderstaande kanalen terugvinden:
	
	\begin{itemize}
		\item Op \href{https://instagram.com/schependom}{Instagram} deel ik zowel mijn sportavonturen als foto's, die ik neem met mijn fototoestel.
		\item Ik probeer ook mijn \href{https://www.linkedin.com/in/vanschependom/}{LinkedIn}-profiel wat up-to-date te houden, een kwestie van \textit{futureproof} (proberen) zijn.
		\item Je kan me ook gewoon mailen: \href{mailto:vincent.vanschependom@student.kuleuven.be}{vincent.vanschependom@student.kuleuven.be}.
	\end{itemize}
	
	Als echte wielerfanaat, ben ik natuurlijk veel bezig met wattages. Daarom hangt de natuurkundige formule voor vermogen in een kadertje boven mijn bed. Ze gaat als volgt: arbeid wordt gedefinieerd door de formule \(W=\int\vec{F}\cdot d\vec{r}\). Nu is het vermogen \(P\) over een tijdsinterval \(T\) gelijk aan:
	
	\[P=\frac{dW}{dT}\] 
	
\end{document}