\documentclass{article}

\usepackage[dutch]{babel}
% Packages voor formules
% Als ge geen optie meegeeft moogt ge usepackage gebruiken met meerdere packages tegelijk
\usepackage{amssymb, amsthm, amsmath}
% We willen Bgtan kunnen typen
\DeclareMathOperator{\Bgtan}{Bgtan}
\usepackage{siunitx}
% We stellen de komma in als decimaalteken
\sisetup{output-decimal-marker={,}}
\usepackage{mhchem}

\title{Joepie, wiskunde!}
\author{Vincent Van Schependom}
\date{Academiejaar 2023-2024}

\begin{document}
	
	\maketitle
	
	\section{Wiskunde-modi}
	
	% We openen inline math met \(
	% We openen display math met \[
	Een rechte door de punten \((a,b)\) en \((c,d)\) met voorwaarde \(a \neq c\) heeft als vergelijking \[y = \frac{(d-b)}{(c-a)}(x-a)+b\].
	
	De formule van Euler (\ref{eq:euler}) werd ontdekt door Euler,
	% We willen later nog kunnen verwijzen naar deze formule, dus we plaatsen het in een equation environment.
	% We hangen er ook een label aan.
	\begin{equation}
		\label{eq:euler}
		1+e^{i\pi}=0.
	\end{equation}
	
	Als we wiskunde gebruiken die over verschillende lijnen loopt, gebruiken we \verb|\begin{align}|. Het linkerlid staat voor het symbool \texttt{\&} en het rechterlid staat erna. De ampersand zorgt ervoor dat het linker- en rechterlid mooi uitgelijnd worden. Er worden standaard labels (2), (3), etc. geplaatst, maar deze kunnen we onderdrukken door een \texttt{*}-symbool toe te voegen aan het begin en einde van de \texttt{align}-environment.
		
	\begin{align*}
		1 & = \sqrt{1} \\
			& = \sqrt{ (-1)(-1) } \\
			& = i \cdot i \\
			& = -1
	\end{align*}
	
	\section{Getallen}
	
	Getallen zien er anders tussen \textit{tekst-modus} en \textit{math-modus}. Als we ze in een \verb*|\num{...}| steken, zullen ze er hetzelfde uitzien. Om een komma als decimaalteken in te stellen, voegen we onder \verb*|\usepackage{siunitx}| nog een extra lijn, namelijk \verb*|\sisetup{output-decimal-marker={,}}|, toe waarin we dit instellen.
	
	We vergelijken hetzelfde getal in de twee modi en daarna een ander (lees: groter) getal in de \verb*|\num{}|-environment:
	
	-1234.56789
	
	\(-1234.56789\)
	
	\num{-1234234.56789}
	
	\section{Roman of Italic}
	
	Er zijn een aantal uitzonderingen op de regel ``als het in \textit{math-mode} moet, moet het cursief''. 
	
	\subsection{Wiskundige objecten van meerdere symbolen}
	
	Namen van wiskundige objecten, die meerdere symbolen bevatten, moeten roman staan en dus niet cursief. Anders wordt de uitdrukking door \LaTeX{} gezien als een product van meerdere wiskundige symbolen.
	
	\(cos x\) is geen juiste notatie voor de cosinus van \(x\), maar \(\cos x\) wel.
	
	\subsection{Nieuwe wiskundige objecten definiëren}
	
	\LaTeX{} kent geen Bgtan, enkel de Engelse variant arctan. We zullen de boogtangens in de preambule van ons document nog moeten definiëren met de lijn \verb*|\DeclareMathOperator{\Bgtan}{Bgtan}|.
	
	\[\Bgtan\]
	
	\subsection{Subscripten}
	
	De massa van de aarde wordt niet mooi weergegeven op volgende manier: \(m_{aarde}\). Om duidelijk te maken aan \LaTeX{} dat het woord `aarde' gewoon tekst is, gebruiken we het commando \verb*|\text{aarde}|.
	\[m_\text{aarde}\]
	
	\subsection{SI-eenheden}
	
	SI-eenheden zetten we met het \texttt{siunitx}-pakket en bijhorend commando \verb*|\si{...}|.
	
	Vergelijk:
	
	 \(kg.m/s^2\)
	 
	 \si{kg.m/s^2}
	 
	 Voor getallen met een eenheid gebruiken we \verb*|\SI{eenheid}{grootheid}|.
	 
	 \SI{9.81}{m/s^2}
	 
	 \subsection{Chemische formules}
	 
	 Voor chemische formules met consistente layouts gebruiken we het \texttt{mhchem}-pakket en bijhorend commando \verb*|\ce{...}|
	 
	 \ce{H_2O}
	 
	 \newpage
	 
	 \section{Formules}
	 
	 \subsection{Accolades en dots}
	 
	 \verb*|\underbrace| zorgt voor de mooie accolade.
	 
	 \verb*|\cdot| zorgt voor een maal-teken.
	 
	 \verb*|\ldots| zorgt voor 3 low dots.
	 
	Verder worden alle spaties in math-modus opgegeten, de spatie na \textit{n} en voor `keer' moet dus in het \verb*|\text| commando toegevoegd worden.
	 
	 \[x^n = \underbrace{x \cdot x \cdot \ldots \cdot x}_{n \text{ keer}}\]
	 
	 \subsection{Limieten}
	 
	 Nu maken we iets met een limiet:
	 
	 \[\lim_{x\to0} \frac{\sin x}{x} = 1\]
	 
	 \(\lim_{x\to0} \frac{\sin x}{x} = 1\)
	 
	 \subsection{Reeksen}
	 
	 \[\sum_{n=0}^{+\infty}\frac{1}{2^n} = 2 \]
	 
	 \(\sum_{n=0}^{+\infty}\frac{1}{2^n} = 2\)
	 
	 \subsection{Integralen}
	 
	 \[ \left(\int_a^x f(t)dt\right)' = f(x) \]
	 
	 \((\int_a^x f(t)dt)' = f(x)\)
	 
	 \section{Matrices}
	 
	 \texttt{tabular} en \texttt{array} zijn qua syntax exact hetzelfde, maar \texttt{tabular} heeft tekst in elke cel en moet in een tekst-omgeving gebruikt worden. \texttt{array}, aan de andere kant, kan in elke cel wiskunde hebben, en moet voorkomen in een math-omgeving.
	 
	\texttt{array} zal nog geen haakjes voorzien voor de matrix, die moeten we zelf toevoegen. Als we ze gewoon rond de environment zetten, zullen ze te klein zijn. Door \verb*|\left| voor het haakje of de vierkante haak te zetten, zal eerst de uitdrukking die na het haakje komt uitgerekend worden, en zal \LaTeX{} daarna de hoogte van het haakje aanpassen naar de hoogte van deze uitdrukking.
	 
	 Gekende notatie:
	 \( \left[ \begin{array}{ll}
	 	a_{11} & a_{12} \\
	 	a_{21} & a_{22}
	 \end{array} \right]\)
	 
	 Determinant:
	\( \left| \begin{array}{ll}
	 	a_{11} & a_{12} \\
	 	a_{21} & a_{22}
	 \end{array} \right|\)
	 
	 Unief-notatie met ronde haakjes:
	 \(\left(\begin{array}{ll}
	 	a_{11} & a_{12} \\
	 	a_{21} & a_{22}
	 \end{array} \right)\)
	 
	 Norm met dubbele streep \verb?\|?
	 	\( \left\| \begin{array}{ll}
	 	a_{11} & a_{12} \\
	 	a_{21} & a_{22}
	 \end{array} \right\| \)
	 
	 \section{Meervoudig functievoorschrift met arrays}
	 
	 Voor de accolade moeten we een \verb*|\| zetten. `Groter dan of gelijk aan' doen we met \verb*|\geq| en `kleiner dan of gelijk aan' met \verb*|\leq|. We hebben enkel een beginaccolade en dus geen eindaccolade. We moeten echter wel zowel \verb*|\left| als \verb*|\right| gebruiken. Aangezien er niets mag volgen na \verb*|\right|, gebruiken we een punt als \textit{placeholer}. Voor de verzameling van de reële getallen gebruiken we \textit{blackboard bold}, dat doen we met het commando \verb*|\mathbb{...}|.
	 
	 \[ |x| : \mathbb{R}\to \mathbb{R} : x \mapsto \left\{  
	 	\begin{array}{ll}
	 		x & \text{als } x \geq 0 \\
	 		-x & \text{als } x<0
	 	\end{array}
	 	\right. \]
	 	
	 	\section{Zelf commando's maken}
	 	
	 	Aangezien sommige commando's zoals \verb*|\mathbb{R}| heel lang zijn om te typen, kunnen we ook zelf commando's ofwel \textit{macros} maken. Dit doen we meestal in de preambule.
	 	
	 	\subsection{Blackboard bold}
	 	
	 	\verb*|\newcommand{\R}{\mathbb{R}}|
	 	
	 	\newcommand{\R}{\mathbb{R}}
	 	
	 	\subsection{Norm van een vector}
	 	
	 	\(  \left\| \vec{v} \right\| \) duurt heel lang om te typen. Daarom kunnen we een nieuw commando (of nieuwe \textit{macro}) definiëren met een argument. We maken eerst duidelijk aan \LaTeX{} hoeveel argumenten we verwachten. Deze argumenten kunnen we vervolgens aanroepen met een \textit{hash}-symbool: 
	 	
	 	\verb?\newcommand{\norm}[1]{\left\| #1 \right\|}?
	 	\newcommand{\norm}[1]{\left\| #1 \right\|}
	 	
	 	\[ \norm{\vec{v}} = 69 \]
	 	
	 	\newpage
	 	\subsection{Partiële afgeleide}
	 	
	 	Dit keer hebben we niet 1, maar twee argumenten.
	 	
	 	\verb?\newcommand{\pd}[2]{\frac{\partial #1}{\partial #2}}?
	 	
	 	\newcommand{\pd}[2]{\frac{\partial #1}{\partial #2}}
	 	
	 	\[ \pd{f(x)}{x} \]
	 	
	 	\section{Voor de rest...}
	 	
	 	We kunnen eigen environments aanmaken met \verb*|\newenvironment|. We kunnen bewijzen, stellingen, eigenschappen, voorbeelden, etc. nummeren met het commando \verb*|\newtheorem|. Er bestaat ook een \verb*|\proof|-environment die het vierkante blokje zet na een bewijs.
	
\end{document}




